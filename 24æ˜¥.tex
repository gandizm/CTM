% Options for packages loaded elsewhere
\PassOptionsToPackage{unicode}{hyperref}
\PassOptionsToPackage{hyphens}{url}
\documentclass[
]{article}
\usepackage{geometry}
\geometry{a5paper, left=0.5cm, right=0.5cm, top=1.5cm, bottom=0.5cm}
\usepackage[UTF8]{ctex}
\usepackage{ulem}
\usepackage{cjkhl}
% \usepackage{color}
% \definecolor{lightblue}{rgb}{.8,.8,1}
% 手动调整,逆天hl
\usepackage{xcolor}
\usepackage{amsmath,amssymb}
%\setcounter{secnumdepth}{-\maxdimen} % remove section numbering
\usepackage{iftex}
\ifPDFTeX
  \usepackage[T1]{fontenc}
  \usepackage[utf8]{inputenc}
  \usepackage{textcomp} % provide euro and other symbols
\else % if luatex or xetex
  \usepackage{unicode-math} % this also loads fontspec
  \defaultfontfeatures{Scale=MatchLowercase}
  \defaultfontfeatures[\rmfamily]{Ligatures=TeX,Scale=1}
\fi
\usepackage{lmodern}
\ifPDFTeX\else
  % xetex/luatex font selection
\fi
% Use upquote if available, for straight quotes in verbatim environments
\IfFileExists{upquote.sty}{\usepackage{upquote}}{}
\IfFileExists{microtype.sty}{% use microtype if available
  \usepackage[]{microtype}
  \UseMicrotypeSet[protrusion]{basicmath} % disable protrusion for tt fonts
}{}
\makeatletter
\@ifundefined{KOMAClassName}{% if non-KOMA class
  \IfFileExists{parskip.sty}{%
    \usepackage{parskip}
  }{% else
    \setlength{\parindent}{0pt}
    \setlength{\parskip}{6pt plus 2pt minus 1pt}}
}{% if KOMA class
  \KOMAoptions{parskip=half}}
\makeatother
\ifLuaTeX
  \usepackage{luacolor}
  \usepackage[soul]{lua-ul}
\else
  \usepackage{soul}
\fi
\setlength{\emergencystretch}{3em} % prevent overfull lines
\providecommand{\tightlist}{%
  \setlength{\itemsep}{0pt}\setlength{\parskip}{0pt}}
\usepackage{bookmark}
\IfFileExists{xurl.sty}{\usepackage{xurl}}{} % add URL line breaks if available
\urlstyle{same}
\hypersetup{
  pdftitle={24春},
  hidelinks,
  pdfcreator={LaTeX via pandoc}}


\usepackage{hyperref}
\begin{document}
\title{24春}
\author{Miaow}
\date{2025.05.21}

\pagestyle{headings}
\maketitle
\tableofcontents
\newpage
%其实我不会写atex

\section{1、中国音乐特点总论}\label{ux4e2dux56fdux97f3ux4e50ux7279ux70b9ux603bux8bba}

\textbf{走进中国音乐}

\begin{itemize}
\item
  音乐对于人的情感不仅能``发散''而且能``净化'',就因为它本身是和谐的,对于人的心灵自然能产生和谐影响。中西神话和历史上都有不少关于音乐感动的传说,城市有借音乐造成的,也有借音乐毁倒的,胜仗有用音乐打来的,重围有用音乐解去的,美人有借音乐取得的,深交有因音乐结成的,名著有从音乐引起思致的,至道有借音乐证成的。

  ------朱光潜《无言之美》
\end{itemize}

\subsection{一、中国音乐与美学思想}\label{ux4e00ux4e2dux56fdux97f3ux4e50ux4e0eux7f8eux5b66ux601dux60f3}

\begin{itemize}
\tightlist
\item
  中国音乐美学始终和中国的哲学、历史、社会、文化融合在一起,构成了以哲学家、文学家、音乐家共同建立的中国音乐美学的复合学科状态。
\item
  以儒家、道家为代表构架了中国音乐美学思想的基础。
\item
  音乐以``和''作为理想的原则,建立了儒家美学思想审音之美的概念。(和合文化)
\end{itemize}

\subsubsection{孔子}\label{ux5b54ux5b50}

\begin{itemize}
\tightlist
\item
  礼乐制度:重视音乐的教育作用-\textgreater 社会秩序、社会风气
\item
  音乐情感表现尺度问题:孔子:儒家中庸之道。乐而不淫、哀而不伤,在情感表达上适度。
\item
  ``君子之音'':文雅敦和、含蓄内敛。
\item
  好的音乐的标准:子谓《韶》:``\textbf{尽美矣,又尽善也。}'' 宫廷雅乐
\item
  谓《武》:``尽美矣,未尽善也。'' 颂扬武功
\end{itemize}

\subsubsection{道家
老子``大音希声''}\label{ux9053ux5bb6-ux8001ux5b50ux5927ux97f3ux5e0cux58f0}

\begin{itemize}
\tightlist
\item
  ``大方无隅,大器晚成,\textbf{大音希声},大象无形。''\,``越好的音乐越悠远潜低''
\item
  好的音乐的境界:悠远潜低(姿态平易)
\item
  有无循环、虚实相间
\item
  John Cage -
  \href{https://www.artnews.com/art-news/news/john-cage-4-33-explained-1234704644/}{4′33″}
  无声音乐-\textgreater 禅宗 ``无中生有''
\end{itemize}

\subsubsection{庄子
返璞归真自然之乐}\label{ux5e84ux5b50-ux8fd4ux749eux5f52ux771fux81eaux7136ux4e4bux4e50}

\begin{itemize}
\tightlist
\item
  三种境界:天籁、地籁、人籁
\item
  顺应并展示人的淳实之性。ou
\item
  心斋坐忘 ``内不觉其一身,外不知乎宇宙,与道冥一,万念俱遣''
\end{itemize}

\sout{何训田-道家音乐 (无关代表作:《阿姐鼓》) }

\subsubsection{《乐记》先秦儒家音乐美学}\label{ux4e50ux8bb0ux5148ux79e6ux5112ux5bb6ux97f3ux4e50ux7f8eux5b66}

\begin{itemize}
\tightlist
\item
  更多涉及音乐的本质问题:``感于物而动,故形于声''\,``是故情深而文明气盛而化神,和顺积而英华发外,唯乐不可以为伪''。-
  内心感触
\end{itemize}

\subsubsection{嵇康《声无哀乐论》,探讨了乐有哀乐的问题。}\label{ux5d47ux5eb7ux58f0ux65e0ux54c0ux4e50ux8bbaux63a2ux8ba8ux4e86ux4e50ux6709ux54c0ux4e50ux7684ux95eeux9898}

\subsubsection{徐上瀛的《溪山琴况》是一部明清时期的琴论专著。主要是从琴乐审美角度提出了自己的琴乐美学见解。}\label{ux5f90ux4e0aux701bux7684ux6eaaux5c71ux7434ux51b5ux662fux4e00ux90e8ux660eux6e05ux65f6ux671fux7684ux7434ux8bbaux4e13ux8457ux4e3bux8981ux662fux4eceux7434ux4e50ux5ba1ux7f8eux89d2ux5ea6ux63d0ux51faux4e86ux81eaux5df1ux7684ux7434ux4e50ux7f8eux5b66ux89c1ux89e3}

况位------情趣、风格

\begin{itemize}
\tightlist
\item
  二十四况为:
\end{itemize}

``和''\,``静''\,``清''\,``远''\,``古''\,``淡''\,``恬''\,``逸''\,``雅''\,``丽''\,``亮''\,``采''\,``洁''\,``润''\,``圆''\,``坚''\,``宏''\,``细''\,``溜''\,``健''\,``轻''\,``重''\,``迟''\,``速''

\begin{itemize}
\tightlist
\item
  以儒、道两家的``淡和''为核心审美观。是中国古代琴艺美学史上一部集大成的琴乐美学理论著作。
\end{itemize}

\subsection{二、音乐和民间传说}\label{ux4e8cux97f3ux4e50ux548cux6c11ux95f4ux4f20ux8bf4}

\begin{itemize}
\tightlist
\item
  内容分类:

  \begin{enumerate}
  \tightlist
  \item
    人物传说
  \item
    地方风物传说
  \item
    史事传说
  \end{enumerate}
\item
  音乐特点:

  \begin{enumerate}
  \tightlist
  \item
    民族性
  \item
    时代性
  \item
    创造性
  \end{enumerate}
\item
  ``有一种非文人的文化传统,它产生于日常生活,而且这种传统也没有专门去培养和发展,是通过口耳相传、互相熏染而自然生成的。''
\end{itemize}

《霸王别姬》《穆桂英挂帅》《秦香莲》《长生殿》《诸葛亮》《梁山伯与祝英台》《萧何月下追韩信》《空城计》等等,民间传说《西厢记》《天仙配》《思凡》《游园·惊梦》《雷峰塔》《白蛇传》

即使同一题材,亦有不同艺术处理和表现手段

\begin{itemize}
\tightlist
\item
  听赏-\textgreater 审美期待
\end{itemize}

\subsection{三、地域音乐元素}\label{ux4e09ux5730ux57dfux97f3ux4e50ux5143ux7d20}

``我曾经在西府走动了两个秋冬,所到之处,村村都有戏班,人人都会清唱。在黎明或者黄昏的时分,一个人独自地到田野里去,远远看着天幕下一个一个山包一样隆起的十三个朝代帝王的陵墓,细细辨认着田项上,荒草中那一截一截汉唐时期石碑上的残字,高高的士屋上的窗口里就飘出一阵冗长的二胡声,几声雄壮的秦腔叫板,我就痴呆了,感觉到那村口的士尖里,一头叫驴的打滚是那么有力,猛然发现了自己心胸中一股强硬的气随同着路上的肌肉疙瘩一起产生了。''

------贾平凹《秦腔》

时代、地域、文化、信仰不同。e.g.南北民歌的区别。《茉莉花》

e.g. 谭维维《华阴老腔一声喊》- 刚直高亢、磅礴豪迈、雄浑、自在随性

e.g. 阿朵 《天生傲鼓》烟子 武术鼓

\begin{itemize}
\tightlist
\item
  独立的艺术表现习惯
\item
  多样独特的音乐文化
\item
  对应的文化景观
\end{itemize}

\subsection{四、声音特征}\label{ux56dbux58f0ux97f3ux7279ux5f81}

八音分类法:\\
金(锣)、石(磬)、丝(二胡)、竹(笛)、土、木(竹枝、响木)、革(鼓)、匏(笙)

金【钟(编钟)、锣、钹】、石(磬)、土(埙)、革【鼓(堂鼓、战鼓等)、鼗(拨浪鼓)】、丝【丝弦发声,琴(古琴)、瑟、筝(古筝)】、木(柷、敔、木鱼)、匏(柷、敔、木鱼)
、竹(笛、箫、排箫)

\begin{itemize}
\tightlist
\item
  ``手挥五弦、游心太虚''、``但识琴中趣,何劳弦上声''的意境
\item
  音色的表现已在音乐构成的意义上,从属于天地老间的立意。
\item
  高频高腔化
\item
  清、尖、亮、透、甜
\item
  意境------枯笔、留白、气口
\end{itemize}

\subsection{五.音乐构成的环境状态}\label{ux4e94.ux97f3ux4e50ux6784ux6210ux7684ux73afux5883ux72b6ux6001}

弦外之音/声

(音乐与环境)共生圈

《印象·丽江》

\begin{itemize}
\item
  ``在中国传统音乐中,有一种西方音乐里见不到的像古琴、昆曲这样的音乐。比如古琴,伴着清泉、鸟鸣,一个人独奏空山,那种似有似无的境界是弹琴人与自然相融之时,音乐已化入非时空。''
\item
  ``最惹眼的是屹立在庄外临河的空地上的一座戏台,模糊在远处的月夜中,和空间几乎分不出界限,我疑心画上见过的仙境,就在这里出现了。这时候船走得更快。不多时,在台上显出人物来,红红绿绿地动,近台的河里一望乌黑的是看戏的人家的船篷。''

  ------鲁迅《社戏》

  民俗、民风
\item
  你见过哪个国家有这么多的地方语言?你有见过哪个地方可以用一种语言唱出声色腔调各异的这么多调来?只有中国的市井戏迷才有这样的福分。这遍布各地的市井戏台上,生旦净末,各类名角儿上演一出出人世间悲欢离合的好戏。``台上一分钟,台下十年功'',唱念做打,字正腔圆,有招有式。台下观众倚着八仙桌,品尝各色点心,上一杯细瓷盖碗茶,品茗听戏看绝活,看到畅快处,大腿一拍,大叫一声``好------''台上台下沸腾一片,那种淋漓尽致的感觉怎一个``绝''字了得。

  -\/-\/-喻学才《古风. 老戏台》

  实景
\item
  赏乐的环境顺应着历史的审美特质,所构成的音响音色变化将音乐厅从单一模式中,走向\textbf{更为多元的文化审美方式},我们不仅仅到音乐厅赏乐,我们还会到更多与音乐音乐的特质相应的场合下去赏乐。
\end{itemize}

\subsection{延伸阅读与拓展欣赏}\label{ux5ef6ux4f38ux9605ux8bfbux4e0eux62d3ux5c55ux6b23ux8d4f}

《中国古代音乐史简编》,夏野著,上海音乐出版社。

《中国音乐词典》,中国艺术研究院音乐研究所《中国音乐词典》编辑部编,人民音乐出版社。

《中国音乐的历史进程与审美》,修海林、李吉提著,中国人民大学出版社。

《神奇秘谱》上、中、下, 中国书店。

《中国民族音乐》,江明惇编著,高校教育出版社。

《土地与歌》,乔建中著,山东文艺出版社。

《古风.老戏台》,喻学才主编,人民美术出版社。

《中唱百年经典20CD》一中国唱片总公司

\section{2、中国音乐的表现方式与内涵}\label{ux4e2dux56fdux97f3ux4e50ux7684ux8868ux73b0ux65b9ux5f0fux4e0eux5185ux6db5}

口头文化传承是一个充满人性活力、有着独特思维的传统,是一个具有真正文化时空含量的传统。民间文化的积淀是超越时空的。文化的大树总是在民间的土壤里埋藏根系,埋藏着生命的古老基因。

------乔晓光 《活态文化------中国非物质文化遗产初探》

\subsection{一、口传心授}\label{ux4e00ux53e3ux4f20ux5fc3ux6388}

``戏路子''

\textbf{师父}、流派

\begin{itemize}
\tightlist
\item
  《春雨杂述·评书》:``学书之法,非口传心授,不得其精。''
\item
  传统音乐根据自己口传心授的规律,不以乐谱的形式而凝固,即不排除即兴性、流动发展的可能,以难以察觉的方式缓慢变化着,是它活力所在。
\item
  ``框泽在曲,色泽在唱''
\end{itemize}

\subsection{二、谱式与符号}\label{ux4e8cux8c31ux5f0fux4e0eux7b26ux53f7}

\begin{itemize}
\tightlist
\item
  文字记谱法
\end{itemize}

vs 拍号、小节线、音高、节奏、框架

\begin{itemize}
\tightlist
\item
  对乐谱再创造
\item
  中国古琴文字谱第一首记载的曲谱《碣石调.幽兰》
\end{itemize}

击鼓谱 vs 字母、唱名、音名

\begin{itemize}
\tightlist
\item
  减字谱
\end{itemize}

左指法徽位,右指法弦序

打谱

\begin{itemize}
\tightlist
\item
  工尺(chě)谱
\end{itemize}

\subsection{三、即兴表演}\label{ux4e09ux5373ux5174ux8868ux6f14}

\begin{itemize}
\tightlist
\item
  西方古典艺术音乐必须严格按照乐谱记载的音高、节奏进行准确的演绎,而中国的谱式只是一个相对固定的框架,更加注重的是个体表现。
\item
  "千变万化"的演绎
\item
  即兴的创造力
\item
\item
  汉族民间歌曲中有两类演变公式是引人注目的,
\item
  一种是以特定的文学母体相维系演变,一种是由某一特定的音乐曲调相维系的民歌群''。
\item
  ``口头文本''的民歌具有一种与生俱来的``鲜活性''
\item
  江南丝竹 合奏音乐
\item
  谱面之外的即兴变化和个性展示
\end{itemize}

\subsection{四、曲体结构}\label{ux56dbux66f2ux4f53ux7ed3ux6784}

\begin{itemize}
\tightlist
\item
  ``艺术形式与我们的感觉、理智和情感生活所具有的动态形式是同构的形式''
\item
  对联(两句)------上、下句结构;绝句(四句)------起承转合四句头结构;律诗(八句)------八板体曲式。''
\item
  单体结构
\end{itemize}

乐段 单二-奏鸣曲式 单三-回旋曲式

音程 \textbf{跳进}级进 加垛

\begin{itemize}
\tightlist
\item
  元.范德现在《诗格》中说道:``作诗有四法:起要平直,承要春容,转要变化,合要渊水。''
\item
  琴曲《阳关三叠》乐曲的四句主要音乐与王维的原诗句式完全相同,在落音和诗句的平仄关系上也极为吻合。
\item
  ``八板''曲牌名曲:琵琶古曲《阳春白雪》、《塞上曲》,江南丝竹《中花六板》,广东音乐《雨打芭蕉》、《饿马摇铃》,\hl{\mbox{山东筝曲《高山流水》}},琵琶曲《青莲乐府》,客家筝曲《出水莲》,民乐合奏《金蛇狂舞》等等都是八板体曲式,并都是由它发展变化而来的。
\end{itemize}

\subsection{五、节奏观念}\label{ux4e94ux8282ux594fux89c2ux5ff5}

\begin{itemize}
\tightlist
\item
  强弱、快慢、松紧
\end{itemize}

乐音 有组织

\begin{itemize}
\tightlist
\item
  在中国传统戏曲、曲艺等音乐中节拍被称为``板眼'',``板''相当于强拍,``眼''相当于次强拍(中眼)或弱拍。
\item
  ``定谱不定音,定板不定腔''
\item
  ``散化的节奏观''
\item
  ``散板不散''
\item
  内在节奏要素
\end{itemize}

关系结合-节拍、重音、休止

周期性、有规律、重复

\section{3、民歌}\label{ux6c11ux6b4c}

风-民歌、雅-贵族之人、颂-功能性

采风-\textbf{民歌艺术}

\subsection{一、我国民歌的发展历史}\label{ux4e00ux6211ux56fdux6c11ux6b4cux7684ux53d1ux5c55ux5386ux53f2}

\begin{itemize}
\tightlist
\item
  黄帝时代的《弹歌》:``断竹,续竹,飞土,逐宍''
\item
  春秋战国时期,我国诞生了第一部诗歌总集《诗经》
\item
  汉唐:民间歌谣不断被吸收进宫廷音乐和文人音乐中。例如:汉代的``相和歌''和唐代的``曲子''
\item
  宋代,为民间歌曲填词的风尚依然盛行,最终出现了``词牌''这一文学形式。
\item
  明清,随着城镇的日渐繁荣和市民文艺的勃兴,民歌迎来了一次历史发展高峰。
\end{itemize}

\subsection{二、民歌与人们的生活}\label{ux4e8cux6c11ux6b4cux4e0eux4ebaux4eecux7684ux751fux6d3b}

\begin{itemize}
\tightlist
\item
  民歌的产生与人们的感情生活密切相关。
\item
  民歌在社会生活中具有许多实际的功用,是社会生活的有机组成部分。
\item
  民歌在人们的交际活动、人生礼仪活动中扮演着重要的角色。
\item
  民歌的社会功用还表现在它具有祭祀与驱邪的功能。
\item
  民歌在促进人们的生产活动中也具有重要的功能。
\end{itemize}

\subsection{三、民歌的艺术特点}\label{ux4e09ux6c11ux6b4cux7684ux827aux672fux7279ux70b9}

\begin{enumerate}
\tightlist
\item
  民歌的歌词大多通俗、易懂,并带有各地民间语言独有的气质魅力。
\end{enumerate}

大山跟里的清泉儿,乌木的瓢儿两舀哩。

睡梦里梦见憨肉儿,清眼泪不由得落哩。

头一帮骡子走远了,第二帮骡子撵了。

阿哥的身子不见了,尕妹的清眼泪淌了。

兰州的白塔固原的钟,拉卜楞寺上的宝瓶。

疼烂了肝花想烂了心,望麻了一对眼睛。

韩湘子出嫁整十八,腰里没勒过带子。

想起来阿哥了碗放下,手抖着抓不住筷子。

------西北``花儿'' 《尕连手令·睡梦里梦见憨肉儿》

\begin{enumerate}
\setcounter{enumi}{1}
\tightlist
\item
  民歌的曲调大多比较短小,材料经济、集中,结构精练,旋律清新流畅,易于记忆。
\item
  民歌的表现手法简洁、朴素,其音乐形象真切、生动。
\end{enumerate}

\subsection{四、汉族民歌音乐艺术}\label{ux56dbux6c49ux65cfux6c11ux6b4cux97f3ux4e50ux827aux672f}

汉族民歌按照音乐特征的不同分为\textbf{号子、山歌、小调}三大体裁类别。

\subsubsection{(一)号子}\label{ux4e00ux53f7ux5b50}

\begin{itemize}
\tightlist
\item
  号子也叫``劳动号子'',是产生并运用于劳动中的民间歌曲。
\item
  《邪许歌》:``今夫举大木者,前呼`邪许',后亦应之。此举重劝力之歌也。
\item
  劳动为民歌提供了最原始的音调和节奏。
\item
  号子的演唱方式多为一领众和。
\item
  在集体劳动中,号子具有双重的功用,首先,它可以鼓舞、调节精神,组织和指挥劳动,具看实用的价值;另一方面,号子也具有一定的艺术表现价值,许多号子的旋律、节奏、结构等方面具有很高的艺术价值,富有极强的艺术感染力。
\end{itemize}

\begin{enumerate}
\tightlist
\item
  搬运号子
\end{enumerate}

\begin{itemize}
\tightlist
\item
  音乐风格较为朴实、粗犷、有力,演唱方式多一领众和。
\item
  作品赏析:\textbf{抬木号子 东北民歌------《哈腰挂》}
\item
  作品赏析:\textbf{挑担号子 四川民歌------《黄杨扁担》}
\end{itemize}

\begin{enumerate}
\setcounter{enumi}{1}
\tightlist
\item
  工程号子
\end{enumerate}

\begin{itemize}
\tightlist
\item
  工程号子音乐在节奏和速度上会产生相应的变化。
\item
  作品赏析:\textbf{湖南常德民歌《打硪歌》}
\end{itemize}

\begin{enumerate}
\setcounter{enumi}{3}
\tightlist
\item
  船渔号子
\end{enumerate}

\begin{itemize}
\tightlist
\item
  船渔号子也相应成为劳动号子中曲调最丰富的一类。
\item
  船夫号子往往包括``起错''、``上滩''、`\,`过滩'\,'、``下滩''等段落,渔民号子一般由``撒网''、``拉网''、``装仓''、``扬帆''等一系列段落组成。
\end{itemize}

多样、复杂、多变

\begin{itemize}
\tightlist
\item
  作品赏析:\textbf{《川江船夫号子》}\\
  这首《川江船夫号子》共有八段。
\end{itemize}

一至三段 ------``平水号子'',\\
第四段 ------``见滩号子'',\\
第五、六段------``上滩号子'',\\
第七段------``拼命号子''\\
第八段------``下滩号子''

打起

号子音乐特征小结:

\begin{itemize}
\tightlist
\item
  直接简朴的表现方法和坚毅粗犷的音乐性格。
\item
  节奏富有律动性。
\item
  音乐材料的重复性,
\item
  常使用领、和相结合的演唱方式。
\item
  曲式结构比较简单,乐段的独立性不强。
\end{itemize}

\subsubsection{(二)山歌}\label{ux4e8cux5c71ux6b4c}

\begin{itemize}
\tightlist
\item
  曲调高亢、自由,节奏悠长。
\item
  主要分布在多山、多丘陵地区,平原地区相对较
\item
  一般山歌、田秧山歌、放牧山歌
\end{itemize}

\paragraph{一般山歌}\label{ux4e00ux822cux5c71ux6b4c}

北方山歌------信天游

加(夹)垛

头-舒展-\textgreater 引子 器乐过门隔 连缀 尾-\textgreater 尾声 活泼 轻快

\begin{itemize}
\tightlist
\item
  ``信天游''主要流行在陕西北部、宁夏及甘肃的东部、山西西部以及内蒙古西南地区。
\item
  信天游的音乐苍茫、辽阔,并带着几分凄凉,是西北高原文化的典型写照。
\item
  信天游一般采用上下句呼应式的单乐段结构。上句的音区通常开阔奔放,高亢热烈,音区跨度大,并达到整首歌曲的最高音,下句则一般是收拢性的乐句,旋律下行,带有较强的叙事性意味。
\item
  作品赏析:\textbf{《蓝花花》、《横山里下来些游击队》}
\end{itemize}

北方山歌------花儿

\begin{itemize}
\item
  ``花儿''也叫``少年'',主要流行于西北的陕甘宁一带
\item
  ``白杨树上上不得,上是树权们挂哩;庄子里`少年'唱不得,唱时老汉们骂哩。
\item
  \textbf{花儿会}
\item
  西北人把花儿的曲调叫做``令'',每个令是一种曲调,总共有上百种。
\item
  歌词有一个特色,就是喜欢运用词组交错结构,形成语言节律的变化。主要有两种形式:
  一是 \emph{\textbf{``头尾齐式''}}, 二称为
  \emph{\textbf{``两担水''或``折断腰''}}。
\item
  ``头尾齐式'':
\item
  由四句唱词(两对上、下句)构成,每句字数大体一致。但是上、下句唱词的词组结构在节奏上形成相互交错的效果,上句最后一个词组由一字或三字构成,下句最后一个词组以两字构成,单双相对。
\item
  例:\\
  打马的 鞭子 闪折了,\\
  走马的 脚步 乱了;\\
  尕妹妹 不像 从前了,\\
  多好的 心思 变了。;
\end{itemize}

上去 高山 望平川,\\
平川里 有一朵 牡丹;\\
看去 容易 摘去难,\\
摘不到 手里是 枉然。

\begin{itemize}
\item
  ``两担水''或``折断腰''
\item
  一种六句式的结构,即在``头尾齐''式的每对上、下句之间,加上一个三至五字的半截句,不仅增加了歌词的内容,而且念起来更加朗朗上口,节奏富于变化。
\item
  例:\\
  三更里月亮懒洋洋,\\
  不开腔,\\
  把月影儿挂在树上;\\
  树叶儿迎风把曲唱,\\
  别惆怅,\\
  阿哥在回家的路上。
\item
  ``花儿''作品赏析\\
  \textbf{甘肃、青海民歌《上去高山望平川》}
\item
  运用了西北的典型音调结构一双四度结构框架(re-sol-la-re)
\item
  旋律音调的进行经历了三次由低音向高音的运动,三起又三落,每次冲到高点(高音sol)都做自由延展拖腔
\item
  自由延长记号
\item
  《白牡丹令》
\end{itemize}

\textbf{北方山歌------山曲:}

\begin{itemize}
\tightlist
\item
  主要流行于山西西北部的河曲、保德、偏关、五寨以及陕北的府谷、神木、内蒙古西南等地。
\item
  山曲大多也是由上、下句乐段构成,音乐材料简洁朴素,常呈合头变尾的旋律关系。
\item
  许多山曲的旋律风格都带有蒙古族民歌的特征。
\item
  走西口的爱情题材。\\
  跳进
\end{itemize}

音程

杀虎口

\textbf{南方山歌:}

\begin{itemize}
\tightlist
\item
  风格比较清丽、婉转而富有意境,表现出南方人的细腻性格。\\
  \textbf{南方山歌------湖南山歌:}
\item
  高腔、平腔、低腔之分
\item
  五句旋律构成的``五句子歌''
\end{itemize}

\textbf{湖南山歌作品赏析:}\\
\textbf{《郎在外间打山歌》}

\begin{itemize}
\tightlist
\item
  平腔山歌
\item
  ``夹垛''
\item
  ``我不晓得是哪处的\textbf{上屋下屋、岭前坳背、巧娘巧爷}生出这样聪明伶俐的患,打出这样\textbf{干干净净、索索俐俐、钻天入地、漂洋过海}的好山歌。
\end{itemize}

\textbf{《板栗开花一条线》}

\begin{itemize}
\tightlist
\item
  多处使用了"顶真格"、"连锁"的旋法进行
\item
  连珠 鱼咬尾 音程 \textbf{音域}
\item
  音域较窄,只有六度
\item
  五句子歌结构
\end{itemize}

\textbf{南方山歌一一四川山歌}

\begin{itemize}
\item
  作品赏析:\textbf{《太阳出来喜洋洋》}
\item
  一字对一音的节奏模式
\item
  平实
\item
  音域不宽,音程进行多半为二度、三度的级进
\item
  句尾多采用自由延长音抒发情感,使音乐舒展悠扬
\item
  \hl{\mbox{作品赏析:\textbf{《槐花几时开》}}}
\item
  作品赏析:\textbf{《跟着太阳一路来》}
\item
  "联八句":即音乐由八句构成,1、2句和7、8句材料相同,形成首尾呼应,旋律风格比较舒展、开阔、高亢。
\item
  中间的四句在是带有念唱性质的"数板"段落
\end{itemize}

\textbf{南方山歌一云南山歌}

\begin{itemize}
\tightlist
\item
  作品赏析:\textbf{《小河淌水》}\\
  旋律大量运用了"顶针格"和回环往复的手法
\end{itemize}

\textbf{南方山歌一一江浙民歌}

\begin{itemize}
\tightlist
\item
  作品赏析:\textbf{《对鸟》}
\end{itemize}

\textbf{放牧山歌:}

\begin{itemize}
\tightlist
\item
  作品赏析:\textbf{《放马山歌》}
\item
  旋律的基本结构为单句体乐段
\item
  乐汇材料采用逐渐减缩的旋律手法
\item
  变化中隐藏着一个有序的数理结构,这赋予了音乐深层次的结构美感
\end{itemize}

\textbf{田秋山歌}

\begin{itemize}
\tightlist
\item
  田秋山歌是以山歌体裁风格为主,又综合有劳动号子和小调的体裁因素。
\item
  作品赏析:\textbf{《拔根芦柴花》}
\end{itemize}

\textbf{山歌音乐特征小结:}

\begin{itemize}
\tightlist
\item
  坦率、直露的表现方法和热情、奔放的音乐性格
\item
  节奏一般自由、悠长
\item
  曲调高元
\item
  结构上,
  汉族地区山歌基本以上、下二句体和四句体为主,其中北方多二句体,南方多四句体个别地区还出现了五句子、联八句等更为复杂的结构形态
\end{itemize}

\textbf{汉族山歌结构中的乐句变化形式}

\begin{itemize}
\tightlist
\item
  自由延长音
\item
  夹垛
\item
  前后加腔
\end{itemize}

\subsubsection{(三)小调}\label{ux4e09ux5c0fux8c03}

\begin{itemize}
\tightlist
\item
  小调又叫
  "小曲"、"小令",主要流行于城镇,用于人们的休息、娱乐、集庆等场合。
\item
  小调分为\textbf{吟唱调、谣曲和时调}三大类。
\item
  \textbf{吟唱调}是小调中实用性功能较强的一个类别,大多与人们的日常生活十分贴近,如其旋律比较接近自然语言形态,多以朗诵为主,拖腔、衬腔少,音乐结构也比较简单、单纯。
\item
  \textbf{谣曲}是人们在日常生活时常演唱的小调,分布很广,数量也很多。它的内容与生活结合紧密,音乐具有浓郁的地方色彩。谣曲相对于吟唱调,结构更加完整,也更具艺术性的加工和考虑。
\item
  \textbf{时调}是小调中流传时间最悠久,传唱范围最广泛的一类。大多十分考究,音乐细腻,表现手法多样,并常配有乐器伴奏,艺术性很强。
\end{itemize}

\paragraph{吟唱调}\label{ux541fux5531ux8c03}

\textbf{儿歌}

\begin{itemize}
\item
  儿歌也就是童谣,是儿童在生活和游戏中演唱的民间小调,儿歌的音域一般比较窄,乐汇变化不多,旋律简洁、生动、活泼。
\item
  作品赏析:\textbf{闽台民歌《天黑黑》}
\item
  音调和节奏简洁、质朴,带有闽台地区民歌的典型气质。
\item
  模进
\item
  变头合尾
\item
  作品赏析:\textbf{江西民歌《斑鸠调》}
\item
  来自于江西南部的客家地区,是一首活泼欢快的歌曲。
\item
  这首童谣后来也成为了赣南地区采茶歌舞、灯彩舞蹈的音乐旋律。
\item
  \textbf{摇篮曲}
\item
  旋律温暖、甜美。速度缓慢。音乐材料朴素简洁
\item
  辽宁民歌《摇篮曲》
\end{itemize}

\paragraph{谣曲}\label{ux8c23ux66f2}

\textbf{诉苦歌}

\begin{itemize}
\tightlist
\item
  作品赏析:\textbf{陕北民歌《揽工调》}
\item
  作品赏析:\textbf{河北民歌《小白菜》}\\
  \textbf{情歌}
\item
  作品赏析:\textbf{四川民歌《康定情歌》}
\item
  作品赏析:\textbf{陕北民歌《三十里铺》}
\item
  作品赏析:\textbf{云南弥渡民歌《绣荷包》}\\
  康定跑马山
\end{itemize}

\textbf{生活歌}

\begin{itemize}
\tightlist
\item
  作品赏析:\textbf{贵州遵义民歌《摘菜调》}
\item
  作品赏析:\textbf{山西民歌《打酸枣》}
\item
  作品赏析:\textbf{东北民歌《看秧歌》}
\item
  切分\\
  \textbf{嬉游歌}
\item
  猜谜语、比心算、讽刺、逗趣。
\item
  作品赏析:\textbf{云南民歌《猜调》}
\item
  作品赏析:\textbf{辽宁民歌《正对花》}
\end{itemize}

\paragraph{时调}\label{ux65f6ux8c03}

\textbf{孟姜女调:}

\begin{itemize}
\tightlist
\item
  孟姜女调也叫
  "春调"、"梳妆台"、"思凡"、"长城调"等,是我国流传最广、影响最大的时调。它的基本旋律形态是流行在江浙一带的小山歌和春调,歌词有"十二月"和"四季"两个版本。
\item
  流传各地的孟姜女调变体大多采用加花的手法进行变化修饰,表现出不同的地域风格。
\item
  作品赏析:\textbf{江苏民歌《孟姜女》}
\item
  作品赏析:\textbf{河北晋县民歌《孟姜女调哭长城》}\\
  孟姜女调基本旋律形态
\end{itemize}

正月 里 来 是 新 春, 家家 户 户 点 红 灯。\\
人 家夫妻 团圆聚,孟姜女的丈夫造长城。

\textbf{茉莉花调:}

\begin{itemize}
\tightlist
\item
  茉莉花调也叫
  "鲜花调",是清代以来十分流行的小曲,它的基本形态是大家所熟知的江苏民歌《茉莉花》。
\item
  普契尼 图兰朵
\item
  作品赏析:\textbf{江苏扬州民歌《茉莉花》}
\item
  作品赏析:\textbf{辽宁长海民歌《茉莉花》}
\end{itemize}

\textbf{剪剪花调:}

\begin{itemize}
\tightlist
\item
  剪剪花又叫
  "剪甸花"、"靛花开"等,是明末清初流行在我国北方的俗曲。它的流传面遍及全国,变体很多,大多风格欢快、喜悦。
\item
  作品赏析:\textbf{河北南皮民歌《放风筝》}
\item
  歌词对姑娘的衣服和手中的风筝都有非常细腻、传神的描写。
\item
  大量衬词的运用使音乐风格特别活泼,富有情趣。
\item
  旋律上很有装饰性,各类倚音、滑音的使用让音调更为华丽。
\end{itemize}

\textbf{绣荷包调:}

\begin{itemize}
\tightlist
\item
  作品赏析:\textbf{山西民歌《绣荷包》}
\item
  作品赏析:\textbf{陕北民歌《绣荷包》}
\end{itemize}

\paragraph{小调音乐特征小结:}\label{ux5c0fux8c03ux97f3ux4e50ux7279ux5f81ux5c0fux7ed3}

\begin{itemize}
\tightlist
\item
  叙事与抒情相交融的表现手法和曲折细腻的音乐性格。
\item
  节奏比较规整,且富有变化。
\item
  旋律乐汇变化丰富,曲折、多形态。
\item
  曲式结构多采用对应式和起承转合式,并发展出多种变化形态。
\end{itemize}

\subsubsection{少数民族民歌艺术}\label{ux5c11ux6570ux6c11ux65cfux6c11ux6b4cux827aux672f}

\paragraph{(一)蒙古族}\label{ux4e00ux8499ux53e4ux65cf}

\begin{itemize}
\tightlist
\item
  蒙古族民歌的体裁一般按照音乐风格的不同,分为\textbf{长调和短调}两大类。
\item
  多声部的 \textbf{``潮尔''唱法},``潮尔''中最特殊是 \textbf{``呼麦''}。
\item
  蒙古族音乐在风格体系上属于中国乐系,因而民歌大多使用五声音阶,偏爱使用羽调式和徵调式。
\item
  旋律风格上,蒙古族民歌的音调大多呈抛物线形,演唱时爱用甩音、倚音等装饰,音程跳动大且频繁,具有开阔、稳健、彪悍的气质。\\
  \textbf{蒙古族民歌赏析:}\\
  \textbf{长调《辽阔的草原》}
\item
  这首民歌的节奏宽广、自由,在每一句的尾部都有长长的拖腔,仿佛让人看到蓝天、白云映衬下的广阔草原。歌曲的旋律极富装饰性,华丽而又深情。此外,大跳音程的使用(例如sol-sol八度大跳、mi-la四度大跳)让音乐显得辽阔、富有张力,具有蒙古人粗犷、豪放的气质。
\end{itemize}

\textbf{短调《嘎达梅林》}

\begin{itemize}
\tightlist
\item
  《嘎达梅林》是一首流传在蒙古草原上的叙事歌曲。讲述的是为蒙古人的土地在战斗中壮烈牺牲的英雄嘎达梅林的故事。这首歌曲为对应式的上下二句体,结构简洁,上句落高音
  la, 下句落低音 la,
  形成相互的呼应。歌曲的音调辽阔而苍凉,起伏很大,充满了悲壮的气势。
\end{itemize}

\paragraph{(二)藏族}\label{ux4e8cux85cfux65cf}

\begin{itemize}
\tightlist
\item
  藏族民歌种类丰富,包括了山歌(牧歌)、劳动歌、爱情歌、风俗歌、诵经歌等类别。其中,以风俗歌中的酒歌、箭歌最具特色。
\item
  转山
\end{itemize}

作品赏析:\textbf{箭歌《姑娘,你往天上听吧》}

\begin{itemize}
\tightlist
\item
  箭歌在藏语中称
  "达鲁",它是从事狩猎活动的射手们夸耀弓箭和箭术时演唱的歌曲,
  歌唱时常常伴随着简单的舞蹈动作,曲调风格清新明快,主要流行于西藏的东南部林区。
\item
  冈仁波齐\\
  \textbf{酒歌《年轻的朋友》}
\item
  酒歌在藏语中称"昌鲁",是喝酒、敬酒时演唱的风俗歌曲,一般伴有简单的舞蹈动作。
\item
  歌曲从山歌式的高亢、自由的旋律进入,显得辽阔、奔放。随后进入歌曲的主体部分,节奏规整,旋律欢畅、明朗,洋溢着节日的气氛。
\end{itemize}

\paragraph{(三)维吾尔族}\label{ux4e09ux7ef4ux543eux5c14ux65cf}

\begin{itemize}
\tightlist
\item
  "十二木卡姆":融民歌、舞蹈、器乐为一体的大型歌舞套曲艺术。
\item
  常运用手鼓、热瓦甫、都它尔、艾杰克、萨巴依等\textbf{特色民族乐器}伴奏。
\item
  维吾尔族民歌在风格上具有波斯一一阿拉伯地区的音乐特点。节奏和节拍上,多切分节奏和弱起节奏。许多民歌都有固定的节奏型,旋法进行曲折、细腻,爱用小幅度的回环反复,犹如锯齿状。
\item
  维吾尔族特色乐器
\end{itemize}

维吾尔族民歌赏析:\textbf{《达坂城的姑娘》}

\begin{itemize}
\tightlist
\item
  这首歌曲运用了维吾尔族舞蹈赛乃姆的节奏,富有动感,在音阶上,它运用了带微分音(微升fa)的七声调式音阶,再加上切分和弱起节奏、旋法上的小幅度回环及乐汇重复进行,具有典型的维吾尔族民歌风味。
\end{itemize}

\textbf{《莱丽古力》}

\begin{itemize}
\tightlist
\item
  歌曲在音乐上的最大特点在于复杂的音阶构成:mi、fa、
  sol、微升sol、升sol、la、升la、si、do、re、升re
\item
  微分音
\end{itemize}

\paragraph{(四)哈萨克族}\label{ux56dbux54c8ux8428ux514bux65cf}

\begin{itemize}
\tightlist
\item
  大多数歌曲带有很强的牧歌特点,民歌的开头普遍有呼唤式的音调,且旋律上扬,乐音多由主音及其四、五度音构成,这一音调也往往是歌曲的核心动机,并由此发展出整首歌曲。
\item
  哈萨克在音乐体系上广泛使用七声自然大小调式音阶。常使用混合节拍(尤其是弹唱歌曲),并且在每小节的节奏划分上形成前短后长的节奏形态,表现出豪放、宽广的音乐性格。
\item
  哈萨克民歌的演唱方式主要有独唱和冬不拉弹唱。
\end{itemize}

\textbf{《美丽的姑娘》《红花》}

\begin{itemize}
\tightlist
\item
  冬不拉弹唱歌曲。它运用了 3 / 8 、 4 / 8 、 5 / 8 、
  6/8相交替的混合拍子,听上去具有一种独特的韵律,效果新颖。
\end{itemize}

\paragraph{(五)朝鲜族}\label{ux4e94ux671dux9c9cux65cf}

\begin{itemize}
\tightlist
\item
  朝鲜族民歌体裁众多,以抒情谣和农谣最具代表性。
\item
  朝鲜族民歌的节奏多使用三拍子,旋律偏爱采用同音重复和小幅度内的环绕式进行,这使得音乐风格具有稳定、端庄、文雅、细腻的特点。
\item
  朝鲜族民歌在演唱时喜欢使用摇声,造成音波的晃动,具有一种激昂的情绪。长鼓、伽倻琴、洞箫、横笛等乐器常作为歌曲的伴奏
\item
  长鼓
\item
  伽倻琴
\end{itemize}

\hl{\mbox{\textbf{阿里郎}}}

\textbf{桔梗摇}

\paragraph{(六)彝族}\label{ux516dux5f5dux65cf}

\begin{itemize}
\tightlist
\item
  各地彝族的民歌在体裁和风格特点上具有不同的特点。
\item
  凉山彝族的叙事歌、婚嫁歌、情歌很有代表性。云南红河地区的彝族流传着大型结构、风格华丽的``四大腔''(海菜腔、山药腔、五山腔和四腔)民歌。贵州的彝族民歌常划分为山上唱的和家里唱的两类。
\item
  彝族歌舞中的常见伴奏乐器有月琴、笛子、三弦等。
\item
  彝族月琴
\item
  彝族三弦
\end{itemize}

彝族民歌赏析:\textbf{《海菜腔》}

\begin{itemize}
\tightlist
\item
  海菜腔是云南彝族 "四大腔"中最为著名的品种。 "四大腔"
  分别流行于彝族尼苏支系的不同区域。
\item
  音乐为大型多段结构的套曲, 多为男女青年在交际习俗活动中演唱。
\item
  常用真假声的转换,多衬词拖腔。
\end{itemize}

\paragraph{(七)侗族}\label{ux4e03ux4f97ux65cf}

\begin{itemize}
\tightlist
\item
  侗族的民歌以其多声部的织体而著称。侗族有一种二声部的民歌叫
  "大歌",侗语叫 "嘎劳","嘎",是歌的意思, ``劳'', 是大的意思。
\item
  大歌根据演唱的内容分为四类:一是鼓楼大歌,二类是叙事大歌,
  三类是儿童大歌, 四类是声音大歌。\\
  作品赏析:\textbf{《嘎吉哟》、\hl{\mbox{《蝉之歌》}}}
\end{itemize}

\section{4、民间器乐艺术}\label{ux6c11ux95f4ux5668ux4e50ux827aux672f}

\begin{itemize}
\tightlist
\item
  我国的民间器乐是指用传统乐器所演奏的民间传统音乐.
\item
  按照表演形式将其分为独奏乐与合奏乐两种。
\item
  清客串(非职业的戏曲演员)、私伙局(佛山民间曲艺)
\item
  独奏乐一般有吹(奏)、拉(弦)、弹(拨)、打(击)数种
\item
  合奏乐则通常分为弦索乐、丝竹乐、吹打乐和清锣鼓四大类。
\end{itemize}

\subsection{发展历史:}\label{ux53d1ux5c55ux5386ux53f2}

\begin{itemize}
\tightlist
\item
  1987年5月在河南舞阳县贾湖新石器时代遗址墓葬中出土的\textbf{骨笛}是实物可考的最古老的乐器。
\item
  浙江余姚河姆渡氏族社会遗址出土的160件\textbf{骨哨}
\item
  \textbf{夏商周}是我国古代乐器的重要发展时期。
\item
  以祭祀功能为主的 "雅乐" 是周代的主要音乐体裁。
\item
  秦汉至魏晋时期,一些\textbf{外族乐器}传入中原,与华夏旧器相互融合,共同组成了日后中华民族传统乐器的庞大家族。
\item
  隋唐五代时期,隋唐宫廷燕乐的曲调集合了来自西域、朝鲜、印度、缅甸等诸多地区的风格色彩。唐代还出现了两个拉弦乐器一一\textbf{轧筝和奚琴}。
\item
  宋、元、明、清是我国民间音乐的大繁盛时期,乐器的种类也得到极大的丰富,
\item
  清末、民国以来,我国的音乐家运用西欧的作曲技术,为传统乐器创作了大量的作品,民间器乐逐步迈入专业化的发展道路。
\end{itemize}

\subsubsection{贾湖骨笛}\label{ux8d3eux6e56ux9aa8ux7b1b}

\subsubsection{河姆渡遗址骨哨}\label{ux6cb3ux59c6ux6e21ux9057ux5740ux9aa8ux54e8}

\subsubsection{商代虎纹大石磬}\label{ux5546ux4ee3ux864eux7eb9ux5927ux77f3ux78ec}

\subsubsection{外族乐器:}\label{ux5916ux65cfux4e50ux5668}

\subsubsection{轧筝和奚琴}\label{ux8f67ux7b5dux548cux595aux7434}

\subsection{独奏乐器及代表作品介绍}\label{ux72ecux594fux4e50ux5668ux53caux4ee3ux8868ux4f5cux54c1ux4ecbux7ecd}

\subsubsection{(一)吹奏乐器}\label{ux4e00ux5439ux594fux4e50ux5668}

\paragraph{1. 笛子}\label{ux7b1bux5b50}

\begin{itemize}
\tightlist
\item
  南方的曲笛和北方的棉笛
\item
  北方的梆笛比较注重舌头的技巧,南方的曲笛更加注重气息的运用。
\item
  北方梆笛曲大多明快、活泼、华丽,代表作品有《喜相逢》、《五梆子》、《扬鞭催马运粮忙》等;
\item
  南方曲笛则更为抒情、含蓄、连贯、优雅,代表作品有《鹧鸪飞》、《姑苏行》等。\\
  作品赏析:
\item
  梆笛代表作品:《五梆子》
\item
  曲笛代表作品:《姑苏行》
\end{itemize}

\paragraph{2. 唢呐}\label{ux5522ux5450}

\begin{itemize}
\tightlist
\item
  簧管乐器, 由哨、气盘、芯子、杆、碗几部分组成。
\item
  音色嘹亮、高亢、雄厚,最适合表现欢快热烈的情绪。著名的唢呐独奏曲有
  \textbf{\hl{\mbox{《百鸟朝凤》}}、《一枝花》}、 《打恵》《将军令》
\end{itemize}

\paragraph{3. 管子}\label{ux7ba1ux5b50}

\begin{itemize}
\item
  \textbf{管子}古代称 "筚篥",是汉代时由西域传入中原的
\item
  管子由管身和哨嘴两部分组成。
\item
  管子最擅长表现的是凄厉、悲凉的情感。
\item
  《通典》载:``筚篥,本名悲篥。 出于胡中,声悲。''
\item
  管曲代表作主要有
  \textbf{《江河水》}、《小二番》、《放驴》、《万年欢》等。
\item
  尺八
\end{itemize}

\paragraph{4. 笙}\label{ux7b19}

\begin{itemize}
\tightlist
\item
  笙是我国古老的吹奏乐器,也是世界上最早使用自由簧的乐器。
\item
  笙以簧、管配合振动发音,能演奏和声。其构造主要由笙簧、笙苗和笙斗三部分构成。
\item
  笙的代表独奏曲有
  \hl{\mbox{\textbf{《凤凰展翅》}}}、《沂蒙山歌》《湘江春歌》等。
\item
  竽和
\end{itemize}

\subsubsection{(二)拉弦乐器}\label{ux4e8cux62c9ux5f26ux4e50ux5668}

\paragraph{1. 二胡}\label{ux4e8cux80e1}

\begin{itemize}
\tightlist
\item
  二胡的历史始于唐朝, 最早发源于我国古代北部地区的少数民族, 史称
  "奚琴"。宋代,更名为 "嵇琴"。
\item
  国乐大师刘天华
\item
  \textbf{二胡的构造}由琴杆、琴轴、琴筒、千斤、琴码、两根琴弦和弓子组成。
\item
  代表作品:刘天华的《病中吟》《良宵》《空山鸟语》《光明行》,华彦钧的《听松》《二泉映月》及《寒春风曲》三首作品。现代作品:《秦腔主题随想曲》《葡萄熟了》《新婚别》《赛马》《战马奔腾》等
\end{itemize}

\hl{\mbox{二胡作品赏析:\textbf{《二泉映月》}}}

\begin{itemize}
\tightlist
\item
  著名民间艺人华彦均(``瞎子阿炳'')代表作
\item
  乐曲共6段,为变奏曲式,表达了艺人对坎坷一生的感慨以及敢于直面生活,积极向上的乐观精神。
\item
  依心曲
\end{itemize}

\textbf{《空山鸟语》}

\begin{itemize}
\tightlist
\item
  刘天华的代表作,是刘天华十大名曲里技巧最难的一首。
\item
  描绘了深山幽谷、百鸟啼鸣的优美意境,是一首极富描绘性的作品。
\end{itemize}

\textbf{《第一二胡狂想曲》}

\begin{itemize}
\tightlist
\item
  当代作曲家王建民的作品,二胡现代风格力作。
\item
  表现了神奇的云南风光、美丽的西双版纳原始丛林以及云南边寨的风土人情、朴实的民风和浓烈的民族音乐风格。\\
  \textbf{《赛马》}
\item
  二胡演奏技巧丰富,展现生动热烈的赛马场面
\end{itemize}

\paragraph{2. 板胡}\label{ux677fux80e1}

\begin{itemize}
\tightlist
\item
  板胡是我国多种梆子腔戏曲和曲艺的主要伴奏乐器。
\item
  板胡的音色明亮、高亢、紧致、激昂,擅长表现热烈、欢快的情绪。
\item
  代表作品有《大起板》、《翻身的日子》、《秦腔牌子曲》、《月牙五更》
\end{itemize}

\paragraph{3.京胡}\label{ux4eacux80e1}

\begin{itemize}
\tightlist
\item
  京剧的主要伴奏乐器
\item
  音色清脆、嘹亮、挺拔、刚健,是一种极富个性的乐器。
\end{itemize}

\textbf{作品赏析:《夜深沉》}

\begin{itemize}
\tightlist
\item
  《夜深沉》是京剧中的器乐曲牌,其音调取自昆曲《尼姑思凡》。
\end{itemize}

\textbf{作品赏析:《梨花颂》}

\paragraph{4.马头琴}\label{ux9a6cux5934ux7434}

\begin{itemize}
\tightlist
\item
  马头琴是蒙古族的特色拉弦乐器,蒙古语称 "绰尔"。
\item
  马头琴音色浑厚温暧、低回婉转,,具有浓厚的草原气息,它常常用于蒙古族长调民歌和舞蹈的伴奏。
\end{itemize}

\textbf{作品赏析:《万马奔腾》、《鸿雁》、《吉祥三宝》}

\subsubsection{(三)弹拨乐器}\label{ux4e09ux5f39ux62e8ux4e50ux5668}

\paragraph{1. 琵琶-批把}\label{ux7435ux7436-ux6279ux628a}

\begin{itemize}
\tightlist
\item
  游牧民族传入的外来乐器。汉代刘熙《释名・释乐器》中有载:"批把本出于胡中,马上所鼓也。推手前曰批,引手前曰把,象其鼓时,因以为名也。"
\item
  南北朝时期,\textbf{曲项琵琶}从波斯经今天的新疆传入我国。
\item
  隋唐时期跃升为宫廷歌舞音乐的重要乐器。
\item
  唐代是琵琶发展的高峰时期
\item
  近代,\textbf{琵琶的形制}基本固定为琴头、琴颈和琴身三个部分。
\item
  代表作有《十面埋伏》、《霸王卸甲》、《阳春白雪》、《月儿高》、《春雨》、《昭君出塞》、《夕阳策鼓》等
\item
  琵琶的传统作品根据风格的不同,可以分为
  \textbf{``文曲''、``武曲''和``大曲''三种}。
\end{itemize}

琵琶琴作品赏析:\\
\textbf{文曲《夕阳策鼓》}

\begin{itemize}
\tightlist
\item
  《夕阳策鼓》曾根据白居易《琵琶行》中的诗句
  ``浔阳江头夜送客''被改名为``浔阳琵琶'',20世纪20年代又被改编为民乐合奏曲《春江花月夜》。
\end{itemize}

\hl{\mbox{\textbf{武曲《十面埋伏》}}}

\begin{itemize}
\tightlist
\item
  《十面埋伏》展现了公元前202年楚汉在垓下决战,汉军用十面埋伏的阵法击败楚军这一震撼人心的古代战争场景,并歌颂了刘邦的威武英姿。
\end{itemize}

\textbf{大曲《阳春古曲》}

\begin{itemize}
\tightlist
\item
  乐曲的主题实际上来自民间器乐曲牌{[}老八板{]},它以活泼轻快的节奏、流畅清新的旋律,表现了大自然冬去春来、生机勃勃的初春景象。
\end{itemize}

\textbf{近代作品《彝族舞曲》}

\begin{itemize}
\tightlist
\item
  创作于20世纪60年代。乐曲以浓郁的民族气息、抒情优美的旋律、淳朴热烈的情感,表现了彝族人民对生活的热爱。
\end{itemize}

\textbf{琵琶当代作品}

林海:《琵琶语》、\hl{\mbox{《欢沁》}}

\paragraph{2. 筝}\label{ux7b5d}

\begin{itemize}
\tightlist
\item
  筝,又称古筝、秦筝。汉代刘熙《释名・释乐器》有云:"施弦高急,筝筝然也"。
\item
  筝的音域宽广,音色清亮,表现力丰富,雅俗共赏。筝在我国南北各地都有流传,并受到各地人们性格、审美趣味等方面的影响,形成了各具特色的风格流派。
\end{itemize}

筝作品赏析:\\
\hl{\mbox{\textbf{山东筝《高山流水・琴韵》}}}

\begin{itemize}
\tightlist
\item
  山东筝音乐风格华丽明快,作品多为宫调式,作品类型分为 "大板曲" 和
  "小板曲"两种,"大板曲"结构较大,使用套曲结构,``小板曲''结构精巧,主要从山东琴书曲牌发展而来。
\item
  《高山流水》是山东筝中 ``大板曲''
  的代表作,也是艺人聚会时必奏的一支曲子,全曲由《琴韵》、《风摆翠竹》、《夜静銮铃》、《书韵》四首曲子组成,每首曲子亦可单独演奏。
\end{itemize}

\paragraph{3. 三弦}\label{ux4e09ux5f26}

\begin{itemize}
\tightlist
\item
  三弦又称``弦子'',元代始称三弦,并成为元曲的主要伴奏乐器。
\item
  \textbf{小三弦音色清亮,多用于昆曲、弹词的伴奏,大三弦音色浑宏低沉,多用于北方鼓书、牌子曲类说唱曲种的伴奏。}
\item
  作品赏析:\textbf{《大起板》、《古城新貌》}
\item
  三弦检造
\item
  大三弦和小三弦
\item
  正仓院螺钿紫檀阮
\item
  阮构造
\item
  大中小阮
\end{itemize}

\subsection{合奏乐及代表作品介绍}\label{ux5408ux594fux4e50ux53caux4ee3ux8868ux4f5cux54c1ux4ecbux7ecd}

\subsubsection{(一) 丝竹乐}\label{ux4e00-ux4e1dux7af9ux4e50}

\begin{itemize}
\tightlist
\item
  丝竹乐是由丝弦乐器和竹管乐器合奏的形式。
\item
  丝竹乐主要流行于南方,代表品种有江南丝竹、福建南音、潮州弦诗、广东音乐、丽江"白沙细乐"等。
\item
  丝竹乐的艺术特点可以概括为"小、轻、细、雅"四个字。
\end{itemize}

\paragraph{1. 江南丝竹}\label{ux6c5fux5357ux4e1dux7af9}

\begin{itemize}
\tightlist
\item
  演奏江南丝竹的组织有 "清客串" 和 "丝竹班" 两种。
\item
  江南丝竹的乐队编制一般7-8人,少则3-5人,二胡和笛子通常为主奏乐器。
\item
  民间乐手在演奏中讲究以即兴的方式相互配合。民间艺决:``胡琴一条线,笛子打打点,洞箫进又出,琵琶筛筛边,双清当板压,扬琴一蓬烟。''
\item
  ``八大名曲'':《中花六板》、《云庆》、《欢乐歌》、《行街》、《慢行街》、《三六》、《慢三六》和《四合如意》。
\end{itemize}

江南丝竹作品赏析:\\
\textbf{《三六》}

\begin{itemize}
\tightlist
\item
  《三六》也叫《梅花三弄》,乐曲的旋律活泼流畅,情绪明朗欢快,具有典型的江南音乐清丽委婉的风格。
\end{itemize}

\textbf{《紫竹调》}\\
据江南民歌《紫竹调》改编而成,曲调风格清新、活泼、优美。

\paragraph{2. 广东音乐}\label{ux5e7fux4e1cux97f3ux4e50}

\begin{itemize}
\tightlist
\item
  广东音乐主要流行于广东省珠江三角洲一带, 大约形成于清末民初。
\item
  广东音乐的早期乐队编制是二弦、提琴(大板胡)、三弦、月琴、笛子,俗称
  ``五架头'',也叫``硬弓组合''。
\item
  ``三件头''------高胡(粤胡)、秦琴和扬琴,叫"软弓组合",之后又增加了洞箫和椰胡,又成为
  "五件头"。
\item
  广东音乐的乐曲大多是表现欢快明亮的情绪以及花鸟、景物,
  有着浓厚的生活气息。代表作有《雨打芭蕉》《鸟投林》《赛龙夺锦》《旱天雷》《步步高》《平湖秋月》等
\end{itemize}

\hl{\mbox{广东音乐作品赏析:\textbf{《旱天雷》}}}

乐曲的旋律大量使用了八度跳进,加上密集而又快速的十六分音符穿梭其间,既有锣鼓喧天的场景刻画,又有幸福喜悦的内心描写,音乐清新优美、朴实硬朗,具有浓厚的岭南风情。

\textbf{《平湖秋月》}\\
近代广东音乐的经典之作

\textbf{《步步高》}

\subsubsection{(二)吹打乐}\label{ux4e8cux5439ux6253ux4e50}

\begin{itemize}
\tightlist
\item
  吹打乐是管弦乐器与打击乐器相配合的合奏形式,民间俗称 "锣鼓" 或
  "鼓吹乐"。
\item
  我国吹打乐类的乐种主要有冀中管乐、山东鼓吹、辽宁鼓乐、西安鼓乐、苏南十番锣鼓、潮州大锣鼓及浙东锣鼓等。
\item
  演奏形式上,吹打乐通常有坐乐和行乐两种。
\end{itemize}

\paragraph{1. 冀中管乐}\label{ux5180ux4e2dux7ba1ux4e50}

\begin{itemize}
\tightlist
\item
  冀中管乐流行于河北省中部,民间称其乐队组织为
  "音乐会"、"吹歌会"。因其擅长吹奏民歌小曲,又称``河北吹歌''。
\item
  冀中管乐有``北乐会''和``南乐会''两种演奏形式。
\item
  冀中管乐较重视管子的表现力,管子的演奏技巧十分丰富。
\end{itemize}

作品赏析:\textbf{《放驴》}

\subsubsection{(三)清锣鼓}\label{ux4e09ux6e05ux9523ux9f13}

\begin{itemize}
\tightlist
\item
  清锣鼓即纯粹以打击乐器合奏的器乐形式。
\item
  清锣鼓的乐器组合形式通分为大、小两种,小型乐队有鼓、小锣、钹和大锣四件乐器。
\item
  其代表乐种有土家族打溜子、陕西``打瓜社''、山西绛州锣鼓等。
\end{itemize}

\paragraph{1. 山西绛州锣鼓}\label{ux5c71ux897fux7edbux5ddeux9523ux9f13}

\begin{itemize}
\tightlist
\item
  山西民间锣鼓被誉为 "中国第一鼓"。绛州锣鼓是当地绛州鼓乐的一部分。
\item
  代表作品:《秦王点兵》、《老鼠娶亲》、《滚核桃》等。
\item
  山西绛州锣鼓有诸多样的敲击方式,并且在演奏中还讲究身段动作的配合。\\
  作品赏析:\textbf{《滚核桃》}
\end{itemize}

\section{5、20世纪中国音乐}\label{ux4e16ux7eaaux4e2dux56fdux97f3ux4e50}

\subsection{20世纪上半叶的中国音乐}\label{ux4e16ux7eaaux4e0aux534aux53f6ux7684ux4e2dux56fdux97f3ux4e50}

\begin{itemize}
\item
  中国的当代文化其实是一种杂交文化,它既不是完全本土的,也不是完全外来的,而是跨文化的。不管是被迫实施门户开放,还是主动进行全面开放,这种杂交文化已是既成事实。从这种意义上说,跨文化研究在中国本土就已经有了充分的资源。这里的杂交概念意味着交织、交叉、互动、文化间多元共存、尊重差异等含义,实际上就是跨文化所包含的意思。

  ------陈纲《文汇报》

  20世纪上半叶的中国音乐
\end{itemize}

\subsubsection{一. 学堂乐歌}\label{ux4e00.-ux5b66ux5802ux4e50ux6b4c}

\begin{itemize}
\item
  《奏定学堂章程》
\item
  音乐启蒙教育,将音乐作为美育方式的重要行为
\item
  作曲家李叔同、沈心工
\item
  代表曲目:
\item
  \hl{\mbox{\textbf{《送别》}:李叔同经典代表之作}}
\item
  \textbf{《春游》}:中国作曲家创作的第一首三部合唱歌曲。
\end{itemize}

\subsubsection{二.
``五四''时期的新音乐}\label{ux4e8c.-ux4e94ux56dbux65f6ux671fux7684ux65b0ux97f3ux4e50}

\begin{itemize}
\tightlist
\item
  五四新文化运动标志着中国现代文化的转型和新发展。
\item
  萧友梅、赵元任、刘天华、黎锦晖等,是中国近现代音乐创作的奠基者。
\item
  ``五四''时期音乐实践的探索精神。
\item
  代表作品:\textbf{《教我如何不想她》------中国艺术歌曲}
\item
  刘天华:在中国器乐音乐文化中,最早探索``从东西的调和与合作之中''去``改造国乐'',提出以``采取本国固有的精粹''去``容纳外来的潮流''的音乐观念。
\item
  代表作品:\textbf{《空山鸟语》}
\end{itemize}

\subsubsection{三.
专业音乐教育的确立}\label{ux4e09-ux4e13ux4e1aux97f3ux4e50ux6559ux80b2ux7684ux786eux7acb}

\begin{itemize}
\tightlist
\item
  蔡元培先生、萧友梅先生于1927年成立了中国第一所专业音乐教育机构------上海国立音专。
\item
  音乐创作的主要特征和表现形式出现了大量采用或借鉴西方艺术歌曲创作模式。
\item
  在中国专业音乐创作的艺术歌曲中,主要强调旋律的抒情性,歌词的隐喻性和平文学性,讲究词与音乐的完美结合。运用西方创作技法表现中国传统音乐元素的器乐作品受到人们的关注。
\item
  代表作品:萧友梅的《问》、\textbf{黄自的《玫瑰三愿》}、青主的《大江东去》、\hl{\mbox{\textbf{贺绿汀的《牧童短笛》}}}
  等。
\end{itemize}

\subsubsection{四.
战争中的歌}\label{ux56db.-ux6218ux4e89ux4e2dux7684ux6b4c}

\begin{itemize}
\tightlist
\item
  20世纪30年代的抗日救亡运动
\item
  音乐的社会功能,音乐的大众化、民族化的语言在战争中得到了极大升华。
\item
  代表人物: 聂耳、冼星海
\end{itemize}

代表作品: \textbf{《铁蹄下的歌女》}

\hl{\mbox{代表作品: \textbf{《黄河大合唱》}}}

\begin{itemize}
\tightlist
\item
  我国近代合唱音乐的一座光辉的里程碑,也是我国近代大型音乐作品的典范之作。
\item
  全曲包括序曲和八个乐章。第一段:黄河船夫曲;第二段:黄河颂;第三段:黄河之水天上;第四段:黄水谣;第五段:河边对口曲;第六段:黄河怨;第七段:保卫黄河;第八段:怒吼吧,黄河。
\end{itemize}

\subsubsection{五. 上海老歌}\label{ux4e94.-ux4e0aux6d77ux8001ux6b4c}

\begin{itemize}
\tightlist
\item
  上海老歌是指诞生于上世纪30年代的黄浦江畔的流行歌曲。
\item
  代表词曲作家: 刘雪庵、黎锦晖、黎锦光、陈蝶衣、严华、陈歌辛等。
\item
  代表作品:\textbf{《毛毛雨》《夜来香》《何日君再来》}、《花样年华》《玫瑰玫瑰我爱你》《香格里拉》《月圆花好》等。
\item
  ``当时霓虹灯通宵不灭,上海滩到处都在播放周璇的歌,家家月圆花好,户户凤凰于飞。''
\end{itemize}

\subsubsection{六.
早期的小提琴音乐}\label{ux516d.-ux65e9ux671fux7684ux5c0fux63d0ux7434ux97f3ux4e50}

\begin{itemize}
\tightlist
\item
  代表人物:马思聪,是中国20世纪著名的小提琴家、作曲家之一,他毕生致力于中西音乐艺术的融合,是中国小提琴音乐早期在中国流传的开拓者。
\end{itemize}

\hl{\mbox{代表作品:\textbf{《思乡曲》}}}

\begin{itemize}
\tightlist
\item
  是我国第一首真正走上国际舞台和第一首被外国小提琴演奏家演奏的小提琴独奏曲。
\end{itemize}

\subsection{20世纪下半叶的中国音乐}\label{ux4e16ux7eaaux4e0bux534aux53f6ux7684ux4e2dux56fdux97f3ux4e50}

\begin{itemize}
\item
  中国音乐不同体裁形式的出现,成为当代中国新音乐创作历史上缤纷的时代。
\item
  大批具有时代特色的中国音乐,使这一时期最具代表性的音乐张扬了现实主义的艺术创作观念。
\item
  ``洋为中用、古为今用''的民族化、大众化的艺术实践,使得传统与西方的融合达到了一个更为现实的开拓。
\end{itemize}

\subsubsection{一.
大型民族管弦乐}\label{ux4e00.-ux5927ux578bux6c11ux65cfux7ba1ux5f26ux4e50}

\begin{itemize}
\item
  大型民族管弦乐队的成立,是建国初期中国音乐一个重要形式的确立。
\item
  20世纪50年代由彭修文组建的中国广播民族乐团,是大型民族管弦乐队做出积极实践与改良的先驱者。
\item
  古曲、器乐独奏曲、民间乐种合奏、传统民歌与歌曲,以及将西方管弦乐队改编为中国民族管弦乐队的大型作品。如《瑶族舞曲》《金蛇狂舞》《娱乐升平》《步步高》《中国狂想曲》《春节序曲》等等。
\end{itemize}

代表作品赏析:\textbf{《瑶族舞曲》}

\begin{itemize}
\item
  原来是一部西方管弦乐队演奏的作品。移植后成为了大型民族管弦乐队常演的作品之一。
\item
  音乐的主题音调来自瑶族的民间音乐,舞蹈性的节奏贯穿全曲。
\item
  以领奏为主、伴奏为辅的主要创作方式,形成了早期大型民族管弦乐主要手段之一。
\end{itemize}

\subsubsection{二. 协奏曲}\label{ux4e8c.-ux534fux594fux66f2}

\begin{itemize}
\item
  小提琴音乐则是20世纪中西交融与互动中,最引人注目的音乐表现形式。
\item
  追求中国传统的音乐元素,借鉴西方创作技法的运动。
\item
  \hl{\mbox{小提琴协奏曲《梁山伯与祝英台》}}、\hl{\mbox{钢琴协奏曲《黄河》}}成为20世纪下半叶中国乐坛上最为经典的乐曲,是中西合璧的典范之作。
\end{itemize}

代表作品赏析:\textbf{小提琴协奏曲《梁山伯与祝英台》}

\begin{itemize}
\tightlist
\item
  分为``呈示部------展开部------再现部''。
\item
  \textbf{\emph{奏鸣曲式}}
\item
  ``爱情''主题、``副部''主题、``抗婚''主题、``楼台会''主题。
\end{itemize}

\textbf{钢琴协奏曲《黄河》}

\begin{itemize}
\tightlist
\item
  全曲分四个乐章。第一乐章《黄河船夫曲》、第二乐章《黄河颂》、第三乐章《黄河愤》、第四乐章《保卫黄河》。
\end{itemize}

\subsubsection{三.
歌剧与舞剧}\label{ux4e09.-ux6b4cux5267ux4e0eux821eux5267}

\begin{itemize}
\item
  中国歌剧与舞剧形式的出现,从创作视野和表现题材都有了极大的拓展,意味着中国音乐在当时已达到了推出综合性艺术水平的能力。
\item
  \textbf{歌剧《江姐》}
  是中国歌剧的优秀作品之一。全剧以江姐演唱的《红梅赞》作为主要音乐主题贯穿发展。
\item
  \textbf{六幕芭蕾舞剧《红色娘子军》}
  是根据同名电影《红色娘子军》改编而成。开辟了西方芭蕾在中国本土化的新形式,也成就了中西文化在芭蕾艺术领域最初探索。
\item
  《快乐的女战士》,《万泉河水清又清》等等音乐,在此后都成为音乐会独立演出的优秀曲目。
\end{itemize}

\subsubsection{四. 现代京剧}\label{ux56db.-ux73b0ux4ee3ux4eacux5267}

\begin{itemize}
\item
  在中国那段特殊的历史时期中,音乐的表达方式与整个社会的发展有着极大的关联。
\item
  八大经典样板戏:《红灯记》《智取威虎山》《沙家浜》《海港》《杜鹃山》《龙江颂》《奇袭白虎团》《盘石湾》。
\item
  经典唱段:\textbf{《迎来春色换人间》《都有一颗红亮的心》《智斗》}
\end{itemize}

\subsubsection{五. 影视音乐}\label{ux4e94.-ux5f71ux89c6ux97f3ux4e50}

\begin{itemize}
\item
  中国拍摄的第一部电影《定军山》
\item
  1930年在上海正式制成中国第一部有声片《歌女红牡丹》。
\item
  20世纪二十到四十年代里,著名电影歌曲:《毕业歌》《渔光曲》《大陆歌》《新女性》《四季歌》《十字街头》《铁蹄下的歌女》《义勇军进行曲》等。
\item
  《马路天使》------\textbf{《四季歌》《天涯歌女》}
\item
  20世纪五六十年代,试图将音乐作为一个独立的电影表现元素,对整部电影有着音乐上的整体思路,参与创作的主体歌曲带有鲜明的民族特色,中华民族的特色。
\item
  代表电影作品:《五朵金花》《刘三姐》《冰上上的来客》《阿诗玛》《护士日记》《上甘岭》《英雄儿女》《南征北战》《红日》《地道战》《柳堡的故事》《铁道游击队》。
\item
  经典影视歌曲:《小燕子》《我的祖国》《花儿为什么这样红》《蝴蝶泉边》\textbf{《弹起我心爱的土琵琶》}
  等。
\end{itemize}

\section{听辨题曲目范围:}\label{ux542cux8fa8ux9898ux66f2ux76eeux8303ux56f4}

\begin{enumerate}
\tightlist
\item
  送别------李叔同 长亭外 古道边
\item
  槐花几时开------南方山歌-四川山歌 高高山上哟 一树喔想哟喂
\item
  \hl{\mbox{十面埋伏}}------弹拨乐器(琵琶)武曲
\item
  黄河大合唱------战争中的歌 光未然作词,冼星海作曲 风在吼
\item
  \hl{\mbox{凤凰展翅}}------笙
\item
  阿里郎------朝鲜族民歌 阿里郎x3哟
\item
  \hl{\mbox{牧童短笛}}------专业音乐教育的确立 钢琴曲 贺绿汀
\item
  \hl{\mbox{二泉映月}}------拉弦乐器(二胡) 华彦均(``瞎子阿炳'')变奏曲式
  依心曲
\item
  蝉之歌------侗族大歌(声音大歌) 听不懂
\item
  \hl{\mbox{百鸟朝凤}}------唢呐独奏曲
\item
  \hl{\mbox{思乡曲}}------20世纪上半叶-早期的小提琴音乐 小提琴独奏曲 马思聪
\item
  \hl{\mbox{高山流水}}------山东筝曲
\item
  \hl{\mbox{黄河钢琴协奏曲}}
\item
  \hl{\mbox{旱天雷}}------民乐-合奏乐-广东音乐 八度跳进
\item
  \hl{\mbox{欢沁}}------琵琶当代作品 林海
\item
  \hl{\mbox{梁祝}}(梁山伯与祝英台)------小提琴协奏曲
\end{enumerate}

\end{document}
